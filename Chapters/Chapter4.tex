% Chapter 4

\chapter{METHODOLOGY} % Main chapter title
%% ANSWER YOUR SPECIFIC OBJECTIVES ONE-BY-ONE!
%% BE SPECIFIC IN METHODOLOGY!

\label{Chapter4} % Change X to a consecutive number; for referencing this chapter elsewhere, use \ref{ChapterX}

\lhead{Chapter 4. \emph{METHODOLOGY}} % Change X to a consecutive number; this is for the header on each page - perhaps a shortened title

%----------------------------------------------------------------------------------------
%	SECTION 1
%----------------------------------------------------------------------------------------


%-----------------------------------
%	SUBSECTION 2
%-----------------------------------

\section{Initial CPU Evaluation}
% To exploit parallelizable steps in the algorithm to further enhance suitability on an FPGA

Successive modifications for speed and output quality will be performed until the improvements meet the target +0.1dB increase in quality and 4x speedup.
Initially, these measurements will be performed in the CPU using MATLAB functions that measure the PSNR, SSIM, and FSIM.

The initial algorithm runs will take place on a computer with an Intel Core i7 3632QM Processor 2.2 GHz, 16GB RAM, and NVIDIA GeForce GT 730M 2GB. 
Profiling the algorithm for bottlenecks will be done with the assistance of MATLAB 2015a Profiler. 
This software package can determine the slowest segments of the algorithm and suggest necessary modifications.
MATLAB will also used to code the algorithms and will assist in the preliminary conversion into a form suitable for the Zynq through its Vision HDL Toolbox.

The rationale behind coding a CPU version of the algorithm is to be able to determine which steps in the algorithm have parallelization potential, while the remaining sequential steps could be easily ported over to the CPU segment of the target SoC.
%----------------------------------------------------------------------------------------
%	SECTION 2
%----------------------------------------------------------------------------------------

\section{FPGA Deployment and Comparative Tests}
Once the criteria are met in the CPU version of the algorithm, the next step is to parallelize using the available FPGA resources.
The platform to be used is the Xilinx Zynq-7000 SoC (embedded CPU+FPGA) on a Digilent Zedboard.
The hybrid FPGA and low-power CPU chip enables the dual advantage of quick sequential processing and fine-grained parallel computing.
The Xilinx Vivado Design Suite 2014.2 software will facilitate the process of converting the CPU-based algorithm to parallel.
In addition, Vivado will be used to fine-tune the algorithm and optimize resource usage in the hardware.




