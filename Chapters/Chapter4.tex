% Chapter 4

\chapter{METHODOLOGY} % Main chapter title

\label{Chapter4} % Change X to a consecutive number; for referencing this chapter elsewhere, use \ref{ChapterX}

\lhead{Chapter 4. \emph{METHODOLOGY}} % Change X to a consecutive number; this is for the header on each page - perhaps a shortened title

%----------------------------------------------------------------------------------------
%	SECTION 1
%----------------------------------------------------------------------------------------

%-----------------------------------
%	SUBSECTION 1
%-----------------------------------
\subsection{Evaluation of state-of-the-art algorithms}
The state-of-the-art algorithm will first be evaluated as to its speed and output quality.
The findings generated will then serve as the baseline for improvement.

%-----------------------------------
%	SUBSECTION 2
%-----------------------------------

\subsection{Modifications for speed and quality}
Successive modifications for speed and output quality will be performed until the improvements are sufficient to guarantee state-of-the-art status.
Initially, these measurements will be performed in the CPU.
This process is iterative.
The criteria are at least +1dB increase in quality and at least 4x speedup.

%----------------------------------------------------------------------------------------
%	SECTION 2
%----------------------------------------------------------------------------------------

\subsection{FPGA Deployment and Comparative Tests}
If the criteria are met in the CPU version of the algorithm, the next step is to parallelize using the available FPGA resources.
A specialized  architecture will be defined

\subsection{Exploiting parallelism for FPGAs}

\section{Data Flow}

\section{Tools to be Used}
The initial algorithm runs will be performed on a computer with an Intel Core i7 3632QM Processor 2.2 GHz, 16GB RAM, NVIDIA GeForce GT 730M 2GB. 
The FPGA to be used is the Xilinx Zynq-7000 SoC (embedded CPU+FPGA) on a Digilent Zedboard.
Using this SoC combines the easy programmability of an embedded CPU with the fine-grained parallelism of an FPGA.
The software consists of MATLAB 2015a and Xilinx Vivado Design Suite 2014.2.