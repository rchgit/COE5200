% Chapter 1

\chapter{INTRODUCTION} % Main chapter title

\label{Chapter1} % For referencing the chapter elsewhere, use \ref{Chapter1} 

\lhead{Chapter 1. \emph{Introduction}} % This is for the header on each page - perhaps a shortened title

%----------------------------------------------------------------------------------------

\section{Background of the Study}
In the digital world, everything is quantized: there are no steps in between digital values, unlike analog.
To approximate analog signals, chips called ADCs (analog-to-digital converters) are employed. 
The fidelity of the output signal of these chips is based on its \textit{resolution}. 
A low-resolution output would mean lower fidelity, but less potential data storage used.
The only way to increase resolution is by increasing the capabilities of hardware. 
In just a few decades, digital images and videos that started as just a small set of pixels (picture elements) has grown into thousands or millions of pixels.
Although today's monitor resolutions have become satisfactory for the human eye, development remains incessant.
The search for ever higher screen resolutions led to the advent of ultra high definition television screens and monitors. 
In the U.S. alone, 80\% of households have at least one high-definition screen \citep{LRG2015}.
However, to date, video sources such as terrestrial transmissions here in the Philippines (cite here) do not use the full resolution of modern televisions. %citation needed

To solve this problem, a class of methods called "super-resolution" are employed.
Super-resolution is the process of rendering or recovering a larger image or video given some low-resolution source \citep{Dong2014}.
Super-resolution finds its applications in diverse fields of study. Examples include video surveillance, in \cite{Caner2003} and \cite{Zhang2010},  medical imaging in \cite{Malczewski2008}, and satellite imaging.
Multi-frame image super-resolution methods use a set of LR images to construct an HR image by exploring the spatial correlations in that set \citep{Cheng2013}.
This kind of SR is applicable in laboratory settings.
In other cases, single frame SR is more appropriate. 
These methods try to extract information from only one HR image, making the task much more difficult than multi-frame.


%----------------------------------------------------------------------------------------

\section{Statement of the Problem}

In many applications such as super-HD (4K) TV, super resolution has to be performed in real time \citep{Shen2014}. However, as noted by \cite{Ishizaka2013} "it is several times slower than real-time" to upscale videos using commodity hardware.
Besides, power consumption is also an important issue. 
To integrate super-resolution processing into existing systems, there must not be a drastic increase in power footprint. 
A number of state-of-the-art methods of SR use GPUs and manycore CPUs, which offer a degree of performance at the expense of heavy power consumption and heat dissipation. 
These current solutions also have unstable frame rates \citep{Wu2011}. 

We are then confronted by the problem of finding a video super-resolution system that uses a fast algorithm to generate high-quality hi-resolution videos from a low-resolution source while maintaining a relatively small power footprint.

A new class of integrated circuits known as SoCs (Systems-on-a-chip), combining an FPGA and a CPU, are demonstrated to have high performance-per-watt ratios. % citation needed
Thus, they are suitable for the constraints specified in this study.
%To date, no published paper uses these recently-developed chips for the purpose of video super-resolution.

%----------------------------------------------------------------------------------------
\section{Objectives of the Study}

\subsection{General Objective}

This study aims to come up with a new video super-resolution algorithm and implement this for use on a hybrid FPGA-CPU (Field Programmable Gate Array - Central Processing Unit) chip.


\subsection{Specific Objectives}

Specifically, the study aims to tackle the following goals:

\begin{enumerate}

	\item To develop a novel real-time video super-resolution algorithm that is capable of at least a +0.1 dB increase in peak signal-to-noise ratio as against other recent algorithms. 
	\item To develop programs to assess the quality of the output of the proposed algorithm and the speed in which it was processed.
	\item To investigate and determine the performance bottlenecks in the developed video super-resolution algorithm.
	\item To investigate the parallelism potential in the algorithm to enhance suitability on the FPGA-CPU hybrid.
	\item To test and validate the FPGA-CPU version of the algorithm and compare its performance with the CPU-only implementation.
	\item To measure and compare the power consumed by the running algorithm on both the CPU and the FPGA-CPU hybrid.
\end{enumerate}

%----------------------------------------------------------------------------------------
\section{Significance of the Study}
It is known that the study of video super-resolution remains in its infancy.
High quality and high speed remain difficult to obtain despite the numerous advances in super-resolution through the years.
Therefore, the study would contribute to the growing number of literature regarding video SR. 	
The results would help in the improvement of picture quality of low-resolution videos on high-resolution monitors (high-definition [1920x1080], quad high-definition [960x540], ultra high-definition[3840x2160]).
%However, the results of the study is not limited to video upscaling. 
%In fact, it may be extended to other applications such as in surveillance.


\section{Scope and Limitations of the Study}

Since previous studies (cited in Review of Related Literature) have shown that clear upscaling is practical only at a factor of 4, it has been decided that this study should concern itself with 4x upscaling only.
Furthermore, the maximum target resolution for this study is 1080p (1920x1080 pixels). 
This is because the hardware interface involved (High Definition Multimedia Interface) only supports up to that resolution.

Two software packages will be used: MATLAB 2015a and Xilinx Vivado Suite 2014.2.
The former is an algorithm prototyping environment and code performance profiler, while the latter is an HDL (Hardware Description Lanuguage) development environment.

