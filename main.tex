%%%%%%%%%%%%%%%%%%%%%%%%%%%%%%%%%%%%%%%%%
% Masters/Doctoral Thesis 
% LaTeX Template
% Version 1.43 (17/5/14)
%
% This template has been downloaded from:
% http://www.LaTeXTemplates.com
%
% Original authors:
% Steven Gunn 
% http://users.ecs.soton.ac.uk/srg/softwaretools/document/templates/
% and
% Sunil Patel
% http://www.sunilpatel.co.uk/thesis-template/
%
% License:
% CC BY-NC-SA 3.0 (http://creativecommons.org/licenses/by-nc-sa/3.0/)
%
% Note:
% Make sure to edit document variables in the Thesis.cls file
%
%%%%%%%%%%%%%%%%%%%%%%%%%%%%%%%%%%%%%%%%%

%----------------------------------------------------------------------------------------
%	PACKAGES AND OTHER DOCUMENT CONFIGURATIONS
%----------------------------------------------------------------------------------------

\documentclass[12pt,oneside,letterpaper]{Thesis} % The default font size and one-sided printing (no margin offsets)
%\usepackage[margin=1in]{geometry}
\graphicspath{{Pictures/}} % Specifies the directory where pictures are stored
\usepackage{times}
\usepackage{graphicx} % Required for including pictures
\usepackage[protrusion=true,expansion=true]{microtype} % Better typography
%\usepackage[opacity=1,scale=1,position={4.2in,-5.6in}]{background}
\usepackage[texcoord]{eso-pic}
\usepackage{titlesec}
\titleformat{\chapter}[block]
{\normalsize\bfseries}{\thechapter.}{10pt}{\normalsize}
\titlespacing*{\chapter}{0pt}{0pt}{0pt}

\titleformat{\section}[block]
{\normalsize\bfseries}{\thesection}{10pt}{\normalsize}
\titlespacing*{\section}{0pt}{0pt}{0pt}

\titleformat{\subsection}[hang]
{\normalsize\bfseries}{\thesubsection}{10pt}{\normalsize}
\titlespacing*{\subsection}{0pt}{0pt}{0pt}

\DeclareMathOperator*{\argmin}{\arg\!\min}

%\usepackage{todonotes}
\usepackage[natbibapa,unnumberedbib]{apacite} % APA Reference Style
\usepackage{pgfgantt} % For gantt chart
\hypersetup{urlcolor=black, colorlinks=true} % Colors hyperlinks in blue - change to black if annoying
\title{\ttitle} % Defines the thesis title - don't touch this

%\backgroundsetup{
%	scale=1,
%%	color=black,
%	opacity=0.4,
%	angle=0,
%	contents={%
%		\includegraphics[width=\paperwidth,height=\paperheight]{Figures/border.png}
%	}%
%}
\AddToShipoutPictureBG{\includegraphics[width=\paperwidth,height=\paperheight]{Figures/border.png}}
\begin{document}

\frontmatter % Use roman page numbering style (i, ii, iii, iv...) for the pre-content pages

\setstretch{1.5} % Line spacing of 1.3

% Define the page headers using the FancyHdr package and set up for one-sided printing
\fancyhead{} % Clears all page headers and footers
\rhead{\thepage} % Sets the right side header to show the page number
\lhead{} % Clears the left side page header

\pagestyle{fancy} % Finally, use the "fancy" page style to implement the FancyHdr headers

\newcommand{\HRule}{\rule{\linewidth}{0.5mm}} % New command to make the lines in the title page

% PDF meta-data
\hypersetup{pdftitle={\ttitle}}
\hypersetup{pdfsubject=\subjectname}
\hypersetup{pdfauthor=\authornames}
\hypersetup{pdfkeywords=\keywordnames}

%----------------------------------------------------------------------------------------
%	TITLE PAGE
%----------------------------------------------------------------------------------------

\begin{titlepage}
\begin{center}

%\textsc{\LARGE \university}\\[1.5cm] % University name
%\textsc{\Large Masteral Thesis Proposal}\\[0.5cm] % Thesis type
\bigskip
\begin{figure}
	\centering
	\includegraphics[scale=0.3]{Figures/dlsulogo.png}
%	\rule{35em}{0.2pt}
%	\caption[]{Conceptual Framework of the Study.}
	\label{fig:dlsulogo}
\end{figure}
%\HRule \\[0.4cm] % Horizontal line
{\ttitle}\\[0.4cm] % Thesis title
%\HRule \\[1.5cm] % Horizontal line


\begin{center}
	
	-------------------\\
	\bigskip
	A Thesis Topic Proposal\\
	Presented to the Faculty of the\\
	Department of Electronics and Communications Engineering\\
	College of Engineering, De La Salle University\\
	\bigskip
	-------------------\\
	\bigskip
	In Partial Fulfillment of\\
	The Requirements for the Degree of\\
	\degreename	\space in \deptname\\
	\bigskip
	-------------------\\
	\bigskip
	by\\
	\bigskip
	\authornames \\
	\bigskip
	\today
	
\end{center}


%\begin{minipage}{0.4\textwidth}
%\begin{flushleft} \large
%\emph{Author:}\\
%{\authornames} % Author name - remove the \href bracket to remove the link
%\end{flushleft}
%\end{minipage}
%\begin{minipage}{0.4\textwidth}
%\begin{flushright} \large
%\emph{Adviser:} \\
%{\supname} % Supervisor name - remove the \href bracket to remove the link  
%\end{flushright}
%\end{minipage}\\[3cm]

%In partial fulfillment of the requirements\\ for the degree of \degreename\\[0.1cm] % University requirement text
%\textit{in}\\[0.1cm]
%\deptname\\[2cm] % Research group name and department name
 
%{\today}\\[4cm] % Date
%\includegraphics{Logo} % University/department logo - uncomment to place it
 
\vfill
\end{center}

\end{titlepage}

%----------------------------------------------------------------------------------------
%	DECLARATION PAGE
%	Your institution may give you a different text to place here
%----------------------------------------------------------------------------------------

% Comment the following 2 lines after finishing the thesis
%\listoftodos
%\clearpage
%\Declaration{
%
%\addtocontents{toc}{\vspace{1em}} % Add a gap in the Contents, for aesthetics
%
%I, \authornames, declare that this thesis titled, '\ttitle' and the work presented in it are my own. I confirm that:
%
%\begin{itemize} 
%\item[\tiny{$\blacksquare$}] This work was done wholly or mainly while in candidature for a research degree at this University.
%\item[\tiny{$\blacksquare$}] Where any part of this thesis has previously been submitted for a degree or any other qualification at this University or any other institution, this has been clearly stated.
%\item[\tiny{$\blacksquare$}] Where I have consulted the published work of others, this is always clearly attributed.
%\item[\tiny{$\blacksquare$}] Where I have quoted from the work of others, the source is always given. With the exception of such quotations, this thesis is entirely my own work.
%\item[\tiny{$\blacksquare$}] I have acknowledged all main sources of help.
%\item[\tiny{$\blacksquare$}] Where the thesis is based on work done by myself jointly with others, I have made clear exactly what was done by others and what I have contributed myself.\\
%\end{itemize}
% 
%Signed:\\
%\rule[1em]{25em}{0.5pt} % This prints a line for the signature
% 
%Date:\\
%\rule[1em]{25em}{0.5pt} % This prints a line to write the date
%}
%
%\clearpage % Start a new page

%----------------------------------------------------------------------------------------
%	QUOTATION PAGE
%----------------------------------------------------------------------------------------

%\pagestyle{empty} % No headers or footers for the following pages
%
%\null\vfill % Add some space to move the quote down the page a bit
%
%\textit{``Thanks to my solid academic training, today I can write hundreds of words on virtually any topic without possessing a shred of information, which is how I got a good job in journalism."}
%
%\begin{flushright}
%Dave Barry
%\end{flushright}
%
%\vfill\vfill\vfill\vfill\vfill\vfill\null % Add some space at the bottom to position the quote just right
%
%\clearpage % Start a new page

%----------------------------------------------------------------------------------------
%	ABSTRACT PAGE
%----------------------------------------------------------------------------------------

\addtotoc{Abstract} % Add the "Abstract" page entry to the Contents

\abstract{\addtocontents{toc}{\vspace{1em}} % Add a gap in the Contents, for aesthetics
It is known that numerous video sources today, such as those recorded by low-resolution devices, have resolutions below that of the monitors that display them.
It is possible to display those videos on high-resolution monitors, however, na\"\i vely resizing the pixels would cause degradation in display quality, hence the need for methods called super-resolution.
Video super-resolution is the process of reconstructing a high-resolution frame sequence from a low-resolution frame sequence while minimizing loss of quality. 
The data for super-resolving signals is always inadequate, therefore, this problem is ill-posed in nature. 
Several algorithms have been proposed in the literature to increase the quality of the high-resolution output, but none of them can perfectly reconstruct a high-resolution frame from a low-resolution frame.
The super-resolution process itself is expensive in terms of computational power.
Worse, there are applications which have real-time constraints, such as video stream upscaling and surveillance, therefore, they require high-performance hardware to meet the time and quality constraints.
This study aims to design and implement a real-time video super-resolution system that uses the Xilinx Zynq-7000 hybrid FPGA-CPU System on a Chip for a relatively low-cost and low-power yet high-performance super-resolution platform.
}

\clearpage % Start a new page

%----------------------------------------------------------------------------------------
%	ACKNOWLEDGEMENTS
%----------------------------------------------------------------------------------------

%\setstretch{1.3} % Reset the line-spacing to 1.3 for body text (if it has changed)
%
%\acknowledgements{\addtocontents{toc}{\vspace{1em}} % Add a gap in the Contents, for aesthetics
%
%I thank sir Carlo Ochotorena for his valuable insights into the super-resolution problem. None of this would be possible without his help\ldots
%}
%\clearpage % Start a new page

%----------------------------------------------------------------------------------------
%	LIST OF CONTENTS/FIGURES/TABLES PAGES
%----------------------------------------------------------------------------------------

\pagestyle{plain} % The page style headers have been "empty" all this time, now use the "fancy" headers as defined before to bring them back

\lhead{\emph{\centering{Table of Contents}}} % Set the left side page header to "Contents"
\tableofcontents % Write out the Table of Contents

\lhead{\emph{List of Figures}} % Set the left side page header to "List of Figures"
\listoffigures % Write out the List of Figures

\lhead{\emph{List of Tables}} % Set the left side page header to "List of Tables"
\listoftables % Write out the List of Tables

%----------------------------------------------------------------------------------------
%	ABBREVIATIONS
%----------------------------------------------------------------------------------------

\clearpage % Start a new page

\setstretch{1.5} % Set the line spacing to 1.5, this makes the following tables easier to read

\lhead{\emph{Abbreviations}} % Set the left side page header to "Abbreviations"
\listofsymbols{ll} % Include a list of Abbreviations (a table of two columns)
{
%\textbf{LAH} & \textbf{L}ist \textbf{A}bbreviations \textbf{H}ere \\
\textbf{GPU} & \textbf{G}raphics \textbf{P}rocessing \textbf{U}nit \\
\textbf{CPU} & \textbf{C}entral \textbf{P}rocessing \textbf{U}nit \\
\textbf{SR} & \textbf{S}uper-\textbf{R}esolution \\
\textbf{PSNR} & \textbf{P}eak \textbf{S}ignal-to-\textbf{N}oise \textbf{Ratio} \\
\textbf{SSIM} & \textbf{S}tructural \textbf{S}imilarity \textbf{I}ndex \textbf{M}easure \\
\textbf{FSIM} & \textbf{F}eature \textbf{S}imilarity \textbf{I}ndex \textbf{M}easure \\
\textbf{FPGA} & \textbf{F}ield \textbf{P}rogrammable \textbf{G}ate \textbf{A}rray \\
\textbf{GPGPU} & \textbf{G}eneral-\textbf{P}urpose computing on \textbf{G}raphics \textbf{P}rocessing \textbf{U}nits\\
\textbf{SoC} & \textbf{S}ystem-\textbf{o}n-a-\textbf{C}hip\\
\textbf{IQA} & \textbf{I}mage \textbf{Q}uality \textbf{A}ssessment\\
\textbf{HVS} & \textbf{H}uman \textbf{V}ision \textbf{S}ystem\\
\textbf{HDL} & \textbf{H}ardware \textbf{D}escription \textbf{L}anguage\\


%\textbf{Acronym} & \textbf{W}hat (it) \textbf{S}tands \textbf{F}or \\
}

%----------------------------------------------------------------------------------------
%	PHYSICAL CONSTANTS/OTHER DEFINITIONS
%----------------------------------------------------------------------------------------

%\clearpage % Start a new page
%
%\lhead{\emph{Physical Constants}} % Set the left side page header to "Physical Constants"
%
%\listofconstants{lrcl} % Include a list of Physical Constants (a four column table)
%{
%Speed of Light & $c$ & $=$ & $2.997\ 924\ 58\times10^{8}\ \mbox{ms}^{-\mbox{s}}$ (exact)\\
%% Constant Name & Symbol & = & Constant Value (with units) \\
%}

%----------------------------------------------------------------------------------------
%	SYMBOLS
%----------------------------------------------------------------------------------------

%\clearpage % Start a new page
%
%\lhead{\emph{Symbols}} % Set the left side page header to "Symbols"
%
%\listofnomenclature{lll} % Include a list of Symbols (a three column table)
%{
%$a$ & distance & m \\
%$P$ & power & W (Js$^{-1}$) \\
%% Symbol & Name & Unit \\
%
%& & \\ % Gap to separate the Roman symbols from the Greek
%
%$\omega$ & angular frequency & rads$^{-1}$ \\
%% Symbol & Name & Unit \\
%}

%----------------------------------------------------------------------------------------
%	DEDICATION
%----------------------------------------------------------------------------------------

%\setstretch{1.3} % Return the line spacing back to 1.3
%
%\pagestyle{empty} % Page style needs to be empty for this page
%
%\dedicatory{For/Dedicated to/To my\ldots} % Dedication text
%
%\addtocontents{toc}{\vspace{2em}} % Add a gap in the Contents, for aesthetics

%----------------------------------------------------------------------------------------
%	THESIS CONTENT - CHAPTERS
%----------------------------------------------------------------------------------------

\mainmatter % Begin numeric (1,2,3...) page numbering

\pagestyle{plain} % Return the page headers back to the "fancy" style
\lhead{}
% Include the chapters of the thesis as separate files from the Chapters folder
% Uncomment the lines as you write the chapters

% Chapter 1

\chapter{INTRODUCTION} % Main chapter title

\label{Chapter1} % For referencing the chapter elsewhere, use \ref{Chapter1} 

\lhead{Chapter 1. \emph{Introduction}} % This is for the header on each page - perhaps a shortened title

%----------------------------------------------------------------------------------------

\section{Background of the Study}
The search for ever higher screen resolutions led to the advent of ultra high definition television screens and monitors. However, to date, common video sources do not use the full capability of most high-resolution televisions of today.

Super-resolution is the process of rendering or recovering a larger image or video given some low-resolution source \citep{Dong2014}.
Super-resolution finds its applications in diverse fields of study. Examples include video surveillance, in \cite{Caner2003} and \cite{Zhang2010},  medical imaging in \cite{Malczewski2008}, and satellite imaging (cite here).
Multi-frame image super-resolution methods use a set of LR images to construct an HR image by exploring the spatial correlations in that set \citep{Cheng2013}.
This kind of SR is applicable in laboratory settings.
In other cases, single frame SR is more appropriate. 
These methods try to extract information from only one HR image, making the task much more difficult than multi-frame.



%----------------------------------------------------------------------------------------

\section{Statement of the Problem}

In many applications such as super-HD (4K) TV, super resolution has to be performed in real time \citep{Shen2014}. However, as noted by \cite{Ishizaka2013} "it is several times slower than real-time" to upscale videos using commodity hardware.
Besides, power consumption is also an important issue. 
To integrate super-resolution processing into existing systems, there must not be a drastic increase in power footprint. 
A number of state-of-the-art methods of SR use GPUs and manycore CPUs, which offer a degree of performance at the expense of electric power.
Current solutions also have unstable frame rates \citep{Wu2011}.
A class of integrated circuits known as FPGAs are demonstrate to have high performance-per-watt ratios.

We are then confronted by the problem of finding a video super-resolution system that uses a fast algorithm to generate high-quality hi-resolution videos from a low-resolution source while maintaining a relatively small power footprint.

%----------------------------------------------------------------------------------------
\section{Objectives of the Study}

\subsection{General Objective}

This study aims to come up with a new video super-resolution algorithm and implement this for use on an FPGA (field programmable gate array).


\subsection{Specific Objectives}

Specifically, the study aims to tackle the following goals:

\begin{itemize}
	\item Improve on the state-of-the-art video super-resolution algorithm in terms of PSNR (peak signal-to-noise ratio) and time complexity. 
	\item Determine the FPGA best suited for the purpose of video super-resolution, considering processing resources and power consumption
	\item Profile the video super-resolution algorithm to determine performance bottlenecks
	\item Exploit parallelizable steps in the algorithm to further enhance suitability on an FPGA
\end{itemize}

%----------------------------------------------------------------------------------------

\section{Scope and Limitations of the Study}

Since previous studies (cite something) have shown that clear upscaling is practical only at a factor of 4, 
it has been decided that this study should concern itself with x4 upscaling only.


%\subsection{Using US Letter Paper}
%
%The paper size used in the template is A4, which is a common -- if not standard -- size in Europe. If you are using this thesis template elsewhere and particularly in the United States, then you may have to change the A4 paper size to the US Letter size. Unfortunately, this is not as simple as replacing instances of `\texttt{a4paper}' with `\texttt{letterpaper}'.
%
%This is because the final PDF file is created directly from the \LaTeX{} source using a program called `\texttt{pdfTeX}' and in certain conditions, paper size commands are ignored and all documents are created with the paper size set to the size stated in the configuration file for pdfTeX (called `\texttt{pdftex.cfg}').
%
%What needs to be done is to change the paper size in the configuration file for \texttt{pdfTeX} to reflect the letter size. There is an excellent tutorial on how to do this here: \\
%\href{http://www.physics.wm.edu/~norman/latexhints/pdf_papersize.html}{\texttt{http://www.physics.wm.edu/$\sim$norman/latexhints/pdf\_papersize.html}}
%
%It may be sufficient just to replace the dimensions of the A4 paper size with the US Letter size in the \texttt{pdftex.cfg} file. Due to the differences in the paper size, the resulting margins may be different to what you like or require (as it is common for Institutions to dictate certain margin sizes). If this is the case, then the margin sizes can be tweaked by opening up the \texttt{Thesis.cls} file and searching for the line beginning with, `$\backslash$\texttt{setmarginsrb}' (not very far down from the top), there you will see the margins specified. Simply change those values to what you need (or what looks good) and save. Now your document should be set up for US Letter paper size with suitable margins.

%\subsection{References}

%The `\texttt{natbib}' package is used to format the bibliography and inserts references such as this one \citep{Reference3}. The options used in the `\texttt{Thesis.tex}' file mean that the references are listed in numerical order as they appear in the text. Multiple references are rearranged in numerical order (e.g. \citep{Reference2, Reference1}) and multiple, sequential references become reformatted to a reference range (e.g. \citep{Reference2, Reference1, Reference3}). This is done automatically for you. To see how you use references, have a look at the `\texttt{Chapter1.tex}' source file. Many reference managers allow you to simply drag the reference into the document as you type.
%
%Scientific references should come \emph{before} the punctuation mark if there is one (such as a comma or period). The same goes for footnotes\footnote{Such as this footnote, here down at the bottom of the page.}. You can change this but the most important thing is to keep the convention consistent throughout the thesis. Footnotes themselves should be full, descriptive sentences (beginning with a capital letter and ending with a full stop).
%
%To see how \LaTeX{} typesets the bibliography, have a look at the very end of this document (or just click on the reference number links).



% Chapter 2

\chapter{REVIEW OF RELATED LITERATURE} % Main chapter title

\label{Chapter2} % Change X to a consecutive number; for referencing this chapter elsewhere, use \ref{ChapterX}

\lhead{Chapter 2. \emph{Review of Related Literature}} % Change X to a consecutive number; this is for the header on each page - perhaps a shortened title

%----------------------------------------------------------------------------------------
%	SECTION 1
%----------------------------------------------------------------------------------------
\section{Digital Image and Video Processing}
Image processing is any form of signal processing in which the input is a still image or a frame of a video. Most image processing algorithms treat the signal as at least a two-dimensional signal. 

The applications of image and video processing nowadays are far reaching. From simple color corrections, geometric transformations, to interpolation, recognition, and even computer vision, image processing algorithms are at the core of these operations.

Digital image processing is the manipulation of digital images by computer algorithms. The preference for the digital domain was spurred by the robustness of digital signals to noise, the relatively easier transformation of signals, and the capability of creating more complex algorithms to solve image processing problems.


\section{Image Quality Assessment (IQA)}
Oftentimes, the quality of the image or video processed must be measured. 
It is rather difficult to scientifically understand quality in terms of human perception.
However quantification of quality is a prerequisite in the development of any image or video processing algorithm.

Currently, there are two primary measures of output image quality, namely, the Peak Signal-to-Noise Ratio (PSNR) and the Structural Similarity Index Measure (SSIM). 
The choice of PSNR or SSIM is typically arbitrary, with a few informal arguments favoring one or the other \citep{Farsiu2004}.
To aid in selection of a suitable metric, an analysis of both image metrics is found in \cite{Hore2010}. 
They state that a mathematical relationship exists between the two metrics, thus making it possible to predict the PSNR from the SSIM and vice-versa. 
They only differ in their sensitivity to image dexgradations as introduced by noise, compression, and hardware limitations.

A recent addition to the list of IQA metrics is the FSIM (Feature Similarity Index Measure) \citep{Zhang2011a}.
The metric was proposed on the basis of human visual systems (HVS) understanding an image mainly due to its low-level features, such as edges and zero-crossings.


\section{Image Super-resolution}

Still-image super-resolution (SR) is the reconstruction of a high-resolution (HR) image given one, or a set of, low-resolution (HR) images. 
In the literature, the words "super-resolution" and "upscaling" are typically interchanged, but \cite{Takeda2009} clarified the distinction between the terms "upscaling" and "super-resolution". 
They stated that "if an algorithm which does not receive input frames that are aliased. it will still produce an output with a higher number of pixels and/or frames (i.e., 'upscaled'), but which is not necessarily 'super-resolved'".
Super-resolution began as the problem of image restoration from a noisy signal \citep{Helstrom1967}.
The first known work that directly tackled SR is that of \cite{tsai1984multiframe}. 

Traditionally, super-resolution of images is performed with several observed LR images, thus being termed "multi-frame SR". 
This is done in order to remove artifacts introduced by the low-resolution camera sensor \citep{Yang2010a}. 
Applications such as medical and satellite imaging prefer the use of multi-frame SR since the minutest of features are crucial to analysis in those fields.

There is another approach which involves only a single observation or image.
The limited set of data severely limits the quality obtainable, thus 
the SR problem becomes ill-posed \citep{Yang2010a}.
Image upscaling for information technology and entertainment is one application that relies on single-frame SR, as there is no available redundancy for the images used in those areas.

Super-resolution is necessary in the following fields of interest:
\begin{enumerate}
	\item Surveillance video: In \cite{Camargo2010}, an SR mosaicking algorithm that stitches and super-resolves UAV (Unmanned Aerial Vehicle) video was implemented on a GPU-CPU pair. They were able to reach as high a PSNR as 41.10 dB for synthetic images. Their application is not real-time, however.
%	\item Satellite imaging: 
	\item Medical imaging: 
%	In \cite{Malczewski2008}, multi-frame SR is accomplished by taking advantage of small spatial shifts in the LR image set.
	\cite{Quan2010} proposed a real-time algorithm that uses localization-based SR, the superior type of SR microscopy for live cell imaging. The algorithm can achieve between 30-500 fps (frames per second) depending on the speed of the camera used.
	\item Video upscaling: This is the main premise of the present study, as with the rest of the papers cited in this chapter.
\end{enumerate}

To the present day, super-resolution remains an active area of research, as evidenced by the wealth of literature cited in this study. 
The following sections present various approaches to SR that rely on several different models.

%---------------
%\subsection{Image Observation Model}
%
%Several factors affect the output of a digital system, including finite aperture size and finite sensor size \citep{Yang2010a}. 
%The image observation model, as adopted by common imaging systems, is shown in Figure \ref{fig:IOM}.
%
%\begin{figure}[ht]
%	\centering
%	\includegraphics[scale=0.5]{Figures/IMAGE_OBSERVATION_MODEL.png}
%	\caption[]{The Image Observation Model.}
%	\label{fig:IOM}
%\end{figure}
%\todo{Explain the image observation model}


\subsection{Frequency Domain and the Nyquist Theorem}
The first SR paper as authored by \cite{tsai1984multiframe} describes the SR process in the frequency domain. 
Their algorithm takes advantage of the shift and aliasing properties of the continuous and discrete Fourier transforms, given a set of multiple shifted low resolution images. 
The main problem is that the typical output of SR methods that depend on Fourier transforms is non-satisfactory for Human Vision Systems.
Ultimately, frequency domain methods have been largely superseded by algorithms which take spatial features into account \citep{Yang2010a}.


\subsection{Sparse Representation Methods}
%\todo{Decide whether the mathematical parts are to be moved to chapter 3}
Those frequency domain methods primarily rely on the Nyquist theorem \citep{Nyquist1928}, which states that any signal can be recovered for as long as it is sampled at a rate at least twice its highest frequency.
That means, to increase resolution in the time domain, sampling frequencies must increase at least twice. 
Unfortunately, this fact gives us two problems. First, to recover a signal with a very high frequency, it may be that it is physically impossible to sample at twice that very high frequency. 
Second, it may be that we are left with too many signal samples, which occupy too much storage space, only to throw much of it later to favor conciseness in representation.
Compression algorithms such as JPEG (Joint Photographic Experts Group) have been employed to reduce the amount of samples required to reconstruct the signal to a degree perceivable enough for its application.
These algorithms rely on \textit{sparse representations}, wherein a signal of length $N$ can be represented in $K<<N$ nonzero coefficients.

%Digital signal data take up too much storage space, but most of this space does not account for the most significant components of the signal it represents.
%Compression and alternative representations are therefore required to reduce storage size while preserving fidelity. 

A recently-established field of study in signal processing, called compressive sensing \cite{Baraniuk2011}, is posed as a new framework for processing signals.
Instead of compressing acquired data, an attempt is made to directly sense the signal in a compressed manner.
	
A highly-related discipline to compressive sensing is dictionary learning. Dictionary learning is the process of training a set of mutually orthogonal basis vectors in order to create a dictionary matrix, with the goal of making representations of similar signals as sparse as possible.  
This matrix can then model any signal as a combination of its columns, better known as "atoms" \citep{Kreutz-Delgado2003}. 

The goal of using dictionary learning in SR is to find a consistent sparse representation of both the LR and HR patches by training an LR and an HR dictionary together. 
This is also known as the sparse-coding process.
%The relative absence of data in the low resolution patches makes it reasonable to consider the LR patch space as a sparse representation of the HR patch space.

\cite{Zeyde2012} proposed an algorithm that uses the Sparseland model previously developed by \cite{Elad2006}. 

\cite{Wright2010} jointly trained a dictionary for low resolution and another for high resolution patches to enforce sparse representation similarity for both patch spaces. Their approach is also robust to noise, as it uses local sparse modeling.
\cite{Yang2012} similarly stressed the importance of learning two coupled dictionaries (observation dictionary and latent dictionary). However, the difference in their methods is that they used one coupled dictionary learning method for single-image SR. 	


%\subsection{Computational Intelligence Methods}
%So far, the previous methods mentioned all have solid mathematical foundations.
%However, it has been found out that a great number of real-world problems cannot be modeled into well-posed mathematical problems, including super-resolution.
%A class of algorithms under "computational intelligence" rely on mimicking natural systems to model and solve such kinds of problems.
%
%\cite{Dong2014} used a deep convolutional neural network in order to learn a mapping between the LR and HR spaces. 
%There are three stages involved: patch extraction and representation, non-linear mapping, and reconstruction.
%Their method is similar to sparse-coding



\section{Challenges in Image SR}
Researchers still struggle with the following challenges, despite significant advancement in the SR literature.


\subsection{Image Registration}
Image registration is the process of mapping two images both spatially and with respect to intensity \citep{Brown1992}.
Image registration is another ill-posed problem, like that of super-resolution
Artifacts caused by registration problems are more noticeable and annoying than the blurring effect as a result of image interpolation \citep{Yang2010a}.
%According to (cite here), image registration is required for the following purposes
\subsection{Computational Efficiency}
According to \cite{Yang2010a}, the computational efficiency of SR is severely limited by the fact that there are a large number of unknowns and computationally-expensive matrix manipulations. 
To alleviate the problem, the author advocates "massive parallel computing".
All of the so-called "real-time SR algorithms" can only handle simple motion models, therefore, they cannot be used for real-world videos.

\subsection{Robustness}
SR algorithms are typically sensitive to signal outliers resulting from motion, blur, noise, etc. 
As the image degradation model parameters are difficult to estimate, researchers nowadays take into account the robustness of their approach \citep{Yang2010a}.

\subsection{Edge preservation}
It is typical in SR algorithms to lose details or edges in the output image/video. 
SR techniques for edge preservation have therefore been proposed. 
\cite{Vishnukumar2014} uses self-examples and high-frequency features to provide edge preservation in SR. Their PSNR goes as high as 30.77 dB, SSIM as high as 0.935, and their highest FSIM is 0.955.
%----------------------------------------------------------------------------------------
%	SECTION 2
%----------------------------------------------------------------------------------------

\section{Video super-resolution}

Video super-resolution is the extension of image SR to moving pictures.
An additional temporal dimension can now be factored in the SR process.
It can generally be divided into two categories: incremental and simultaneous \citep{Su2011}.
The former category is faster but less visually consistent to the human eye.
\cite{Liu2014} mentions that video SR is relatively more challenging than image SR which has been studied for decades, due to the presence of an additional temporal dimension.


\subsection{Bayesian Methods}

\cite{Liu2014} propose a Bayesian video SR system that can simultaneously estimate the HR frame, motion flow fields, blur kernel, and noise level.
Their method works best when the motion is slow and smooth, and would fail if there are significant lighting changes and occlusion.
They acknowledged that aliasing both benefits and "derails" super-resolution.

%-----------------------------------
%	SUBSECTION 2
%-----------------------------------

\subsection{Computational Intelligence Methods}
As in image SR, video SR is a highly nonlinear task and is amenable to processing via computational intelligence. 
For example, \cite{Cheng2013} constructed an artificial neural network (ANN) for video SR.
It is a four-stage algorithm, consisting of classifiers to categorize the image, motion-trace volume collection for pixel tracking, temporal adjustment for fast motions and complicated scenes, and ANN learning.

\section{High Performance Computing Platforms}
\cite{Yang2010a} suggest that high-performance hardware matters in tackling super-resolution problems. 
Typically, image SR algorithms are first developed for computer CPUs.
Modern CPUs (central processing units) of computers combine high-frequency processors with a degree of parallelism to add more processing power to algorithms.
Even so, the CPU is not enough to handle tasks such as SR in real-time.
There are several steps in the SR process that may be implemented as parallel tasks.
Following are the discussions on GPUs, manycore coprocessors, and FPGAs, three parallel platforms commonly in use today.

\subsection{Graphics Processing Units}
GPUs (Graphics Processing Units) have been favored in recent years for this task, as it offers high amounts of parallelism (due to its multiple cores) and compatibility with existing computer systems and programming paradigms.
The two major discrete GPU vendors, NVIDIA and AMD, provide specialized tools to offload massively parallel tasks, a process known as GPGPU (General-Purpose computing on GPU).

GPUs are classified as stream processors, because their architecture makes use of a minimal kernel program that processes all data input (the stream). 
This enables GPUs to process a large amount of data in parallel. 

\cite{Wu2011} claims 6x speedup against the same algorithm implemented on a CPU. 
%\cite{Shen2014} used a real-time learning-based SR algorithm based on error feedback. 

\subsection{Manycore Coprocessors}
This class of parallel processors are based off CPU architectures but have more cores than the traditional CPU and are meant to run at a lower frequency. 
A host CPU passes the appropriate parallel instructions to the manycore coprocessor and subsequently fetches the results of the computation.
Manycore processors offer more programmability than GPUs simply by the fact that they share the same architecture as the host CPU. 
The only known product in this category is that of Intel MIC (Many Integrated Core) architecture \citep{Intel2014}.

\cite{Ishizaka2013} demonstrated a power-efficient real-time SR system that uses a virtual pipeline to improve the performance as well as the utilization of both the manycore and the host processors. 
Their set-up was able to achieve 31.5 fps, satisfying the real-time requirement.
The downside with their setup is the limited adoption of the MIC platform and the power requirement of about 300 W \citep{Intel2014}.


\subsection{Field Programmable Gate Arrays (FPGAs)}
FPGAs (Field Programmable Gate Arrays) are logic devices that can be reconfigured by a designer on the field after being manufactured.
The most basic unit of an FPGA is the CLB (Configurable Logic Block)
It consists of three main elements, namely, the LUTs (look-up tables), the flip-flops, and the routing matrix.
LUTs act as programmable "logic gates", being able to model any Boolean operation possible. 
Flip-flops are designed to temporarily store LUT output and to facilitate correct timing of sequential logic processes. 
The routing matrix connects CLBs together as necessitated by the overall design.

\begin{figure}
	\centering
	\includegraphics{Figures/fpga_block.jpg}
	\label{fig:fpga_block}
	\caption[]{An FPGA Configurable Logic Block (CLB)}
\end{figure}

Since at the lowest level, logic circuits are inherently parallel and real-time, FPGAs offer optimization potential that cannot be realized when using instruction-based platforms such as CPUs and GPUs. FPGAs typically run at much lower frequencies than CPUs and GPUs, making them more power efficient.
Higher-end FPGAs even offer the ability to be partially dynamically reconfigured, so that even while it is running, parts of the FPGA fabric gets their design altered \citep{Dye2012}.
These factors makes FPGAs suitable for the most computationally-intensive real-time applications while conserving energy.
	
\cite{Sirowy2008} investigated the reasons why an FPGA offers high speedups over sequential processing devices such as CPU, manycore, and GPU. 
Among these are: the removal of an instruction fetch step, hiding the control instructions, executing multiple instructions in parallel, and pipelining instructions.


\subsection{Design Considerations and Strategies}
Since the SR system of this study is to be integrated into other computing systems, it is imperative to develop an embedded system, one which consumes less energy.
The more power used, the more heat is generated. According to \cite{Anderson2003}, failure rates double for every 15 degree Celsius rise in temperature.
In this light, for an embedded system, the GPU and CPU are not applicable processors.
\cite{Mittal2014} considered using an "unconventional core" such as an FPGA to realize lower power-consumption in an embedded system.
%\cite{Struyf2014}
%\todo{Finish citation of Struyf}

\section{Comparison of CPU, GPU and FPGA}

\cite{Asano2009} compared the performance of the CPU, GPU and FPGA in image processing applications. 
They noted that CPUs are consistently lagging behind the GPU and FPGA, while the GPU is best for "naive computation methods" in which processing takes place on a per pixel basis.

\cite{Fowers2012} compared the performance and the energy expended by FPGAs, GPUs and multicore CPUs. 
This paper is significant to the present study since their focus is on sliding-window algorithms, which take the data on a per-block basis instead of per-pixel. This makes computation more efficient.

The following figure illustrates the difference between a sequential processor (CPU) and the FPGA. 
In the CPU, both data and instructions are being fetched from memory. 
These instructions are then interpreted into functions to be able to process the data input.
Then the intermediate results are then stored in the same memory.
On the other hand, the FPGA does not have any instruction to fetch from memory, since all the functions are already defined as hard-wired circuitry. 
At the same time, the FPGA can have as many parallel functions as possible, and that data may be pushed into the next function stage in just one clock tick.

\begin{figure}[!ht]
	\label{fig:CPUvsFPGA}
	\centering
	\includegraphics[scale=0.6]{Figures/CPUvsFPGA.png}
	\caption[]{Comparison of CPU and FPGA working pipelines \citep{Flynn2012}.}
\end{figure}

\subsection{Use in super-resolution applications}
The following papers prove the feasibility of an FPGA in SR applications.
\cite{Angelopoulou2009} created a real-time video SR system on an FPGA that is robust against noise.
It uses the iterative back projection algorithm. 
However, the system depends on an adaptive image sensor 
\cite{Szydzik2011} constructed a high quality SR system on an FPGA. 
They were able to achieve 2x upscaling at 25 fps while using only less than 37\% of FPGA resources of the state-of-the-art algorithm at that time.
\cite{Bowen2008} achieved 3x speedup over equivalent software (CPU) implementations.

%\subsubsection{As video processors}
%\cite{Roth2011} used low-cost FPGA hardware to accomplish real-time video processing tasks such as deinterlacing, alpha blending, and frame buffering.

\section{FPGA-CPU System-on-a-chip}

The Zynq-7000 programmable System-on-a-Chip is the first-of-its-kind FPGA-CPU hybrid available market.
It combines the high-performance, low-power ARM Cortex A9 CPU, high-end computer components such as DDR3 RAM, and a programmable logic fabric similar to that of an FPGA.
One advantage of using this SoC is that several peripherals are already built-in and hardwired, eliminating the need for soft IP cores for common devices.
Another is that of faster sequential processing. 
According to Amdahl's Law, the speedup of a parallel program is limited by the time needed for the sequential fraction of the program \citep{Amdahl1967}.
Having both a highly parallel programmable fabric and a fast sequential processor is a great help if an algorithm has to run real-time. 
% Chapter 3

\chapter{METHODOLOGICAL FRAMEWORKS} % Main chapter title

\label{Chapter3} % Change X to a consecutive number; for referencing this chapter elsewhere, use \ref{ChapterX}

\lhead{Chapter 3. \emph{Methodological Frameworks}} % Change X to a consecutive number; this is for the header on each page - perhaps a shortened title

%----------------------------------------------------------------------------------------
%	SECTION 1
%----------------------------------------------------------------------------------------



%\begin{figure}[htbp]
%	\centering
%	\includegraphics{Figures/framework.pdf}
%	\rule{35em}{0.2pt}
%	\caption[Conceptual Framework]{Conceptual Framework of the Study.}
%	\label{fig:Framework}
%\end{figure}



%-----------------------------------
%	SUBSECTION 1
%-----------------------------------
\section{Image Quality Assessment (IQA)}

For image quality to be assessed in the study, three metrics will be used.
They are the peak signal-to-noise ratio (PSNR), the structural similarity index measure (SSIM), and the feature similarity index measure (FSIM). The PSNR measures the robustness the algorithm against noise found in the LR image.
The SSIM measures how similar two images are to one another, ACCO

Given a reference image \textit{a} and a test image \textit{b}, both of the same size \textit{M}x\textit{N},the PSNR is computed using the following equation \citep{Hore2010}:

\begin{equation}
PSNR(a,b) = 20 \log_{10}\frac{255}{\sqrt{MSE(a,b)}}
\end{equation}

where the MSE (mean squared error) is computed as follows:
\begin{equation}
MSE(a,b) = \frac{1}{MN} \sum\limits_{i=1}^{M} \sum\limits_{j=1}^{N} (a_{ij}-b_{ij})^2
\end{equation}


The SSIM is computed using the following equation \citep{Hore2010}:

\begin{equation}
SSIM(a,b) = l(a,b) c(a,b) s(a,b)
\end{equation}

where
\begin{eqnarray}
l(a,b) = \frac{2\mu_a\mu_b+C_1}{\mu_a^2+\mu_b^2+C_1} \label{eqn:luminance_ssim}\\
c(a,b) = \frac{2\sigma_a\sigma_b+C_2}{\sigma_a^2+\sigma_b^2+C_2}\label{eqn:contrast_ssim}\\
s(a,b) = \frac{\sigma_{ab}+C_3}{\sigma_a\sigma_b+C_3}\label{eqn:structure_ssim}
\end{eqnarray}

Equation \ref{eqn:luminance_ssim} measures the similarity in luminance and is equal to 1 only if $\mu_a=\mu_b$.
Similarly, equation \ref{eqn:contrast_ssim} compares the standard deviation of the two images (which corresponds to the contrast). 
It will only equal to 1 if $\sigma_a=\sigma_b$.
The structure comparison equation (\ref{eqn:structure_ssim}) measures the correlation between the images using the covariance $\sigma_{ab}$ between them.

The FSIM is computed using the following equation \citep{Zhang2011a}:

\begin{equation}
\frac{\sum_{x\in\Omega}S_L(\mathbf{x})\cdot PC_m(\mathbf{x})}{{\sum_{x\in\Omega}PC_m(\mathbf{x})}}
\end{equation}

where $\Omega$ is the whole spatial image domain and $S_L$ is the SSIM.
$S_L(\mathbf{x})$

\section{Algorithm Framework}

% Explain the framework
Figure \ref{fig:algoframe} summarizes the flow of the proposed algorithm.
The major components of this algorithm are the deblurring module, the temporal consistency module, and the edge-preservation module.

The deblurring module removes image blur caused by motion and the camera sensor.
To accomplish the process of deblurring, a blur kernel is first estimated.
Motion blur is usually modeled as
\begin{equation}
	B = K * L + N	
\end{equation}
where $B$ is the blurred image, $K$ is the motion blur kernel, $N$ is unknown noise introduced during image acquisition, and $*$ is the convolution operator \citep{Cho2009}.

To be able to deconvolve the $K$ and $L$ terms, both of these terms must be optimized in an alternating fashion.
\begin{equation}
L' = \argmin_L{(||B-K*L||+\rho_L(L))}
\end{equation}
\begin{equation}
K' = \argmin_K{(||B-K*L||+\rho_K(K))}
\end{equation}

The $\argmin$ notation denotes an operation to minimize the variable subscripted in the $\argmin$ subject to the constraints stated inside the parentheses.
The symbols $||$ represent a norm, a function that assigns a strictly positive length to each vector in a vector space except the zero vector.
It is typical to use the 2-dimensional, or $L_2$, norm, as it is the intuitive notion of length in any n-dimensional space, and is computationally efficient.
The additive term at the end of the equations are regularization terms.
This is added to prevent overfitting of the data or to provide additional information to solve an ill-posed problem.

The temporal consistency module ensures that successive video frames are consistently super-resolved with respect to time. 
It uses the preceding HR frame to accomplish this task.
Temporal consistency is computed as follows:

\begin{equation}
	C_t = exp(-\frac{1}{2\rho_0^2}||M_t^{t-1}X_t-\widetilde{X}_{t-1}||^2)
\end{equation}

where $X_t$ is the low-resolution frame at time $t$, $\widetilde{X}_{t-1}$ is the preceding high-resolution frame, $M_t$ is the motion matrix, and
\begin{equation}
\rho_0^2 = \rho \cdot var(B(\widetilde{M}_t^{t-1}\bar{X}_t-\widetilde{X}_{t-1} = 1))
\end{equation}
is the $l_0$ norm of the difference between the motion-blurred current frame and the previous high-resolution frame.

The edge-preservation module takes the finer details of the LR video frames and interpolates the HR version of the edges. 
This will be accomplished through weighted least-squares filtering \citep{Farbman2008}.

Given an image $g$ , we are attempting to find $u$ such that $u$ is as close as possible to $g$ while being smooth as possible everywhere, except across significant gradients in $g$. 
This is formalized as minimizing the equation

\begin{equation}
	(u-g)^T(u-g)+\lambda(u^T D_x^T A_x D_x u + u^T D_y^T A_y D_y u)
\end{equation}

where $A_x$ and $A_y$ are diagonal matrices containing the smoothness weights, and the matrices $D_x$ and $D_y$ are discrete differentiation operators.

A final reconstruction step incorporates output from all the three major modules to generate the HR frame sequence.

The three modules were built in parallel to each other so that the algorithm would be amenable to optimization on the FPGA side of the SoC.

% Place figure here
\begin{figure}[!ht]
	\centering
	\includegraphics[scale=0.7]{Figures/ALGO_FRAMEWORK.png}
	\caption[]{Framework for the Proposed Algorithm.}
	\label{fig:algoframe}
\end{figure}


%-----------------------------------
%	SUBSECTION 2
%-----------------------------------


\section{Hardware Framework}
Figure \ref{fig:hardframe} illustrates the flow of data across the hardware devices to be used in the study.
A video source such as a camera or prerecorded file will be sent for super-resolution on the SoC, which will then display the result on the monitor in real-time.
Initially a conventional computer will serve as the development and evaluation  environment for the SR algorithm.
This computer contains the MATLAB and Vivado software packages that are used for the development and testing of the algorithm in both the computer and the FPGA-CPU hybrid.
A version of the algorithm can then be sent to the SoC for testing and fine-tuning.

\begin{figure}[!ht]
	\centering
	\includegraphics[scale=0.7]{Figures/HARDWARE_FRAMEWORK.png}
	\caption[]{Framework for Hardware Implementation.}
	\label{fig:hardframe}
\end{figure}


% Explain the framework



%----------------------------------------------------------------------------------------
%	SECTION 2
%----------------------------------------------------------------------------------------

% Chapter 4

\chapter{METHODOLOGY} % Main chapter title
%% ANSWER YOUR SPECIFIC OBJECTIVES ONE-BY-ONE!
%% BE SPECIFIC IN METHODOLOGY!

\label{Chapter4} % Change X to a consecutive number; for referencing this chapter elsewhere, use \ref{ChapterX}

\lhead{Chapter 4. \emph{METHODOLOGY}} % Change X to a consecutive number; this is for the header on each page - perhaps a shortened title

%----------------------------------------------------------------------------------------
%	SECTION 1
%----------------------------------------------------------------------------------------

%-----------------------------------
%	SUBSECTION 1
%-----------------------------------
\section{Benchmarking of state-of-the-art algorithms}

The metrics to be tested are the execution time, the PSNR, SSIM, and FSIM.

Given a reference image \textit{a} and a test image \textit{b}, both of the same size \textit{M}x\textit{N},the PSNR is computed using the following equation \citep{Hore2010}:

\begin{equation}
PSNR(a,b) = 20 \log_{10}\frac{255}{\sqrt{MSE(a,b)}}
\end{equation}

where
\begin{equation}
MSE(a,b) = \frac{1}{MN} \sum\limits_{i=1}^{M} \sum\limits_{j=1}^{N} (a_{ij}-b_{ij})^2
\end{equation}

The SSIM is computed using the following equation \citep{Hore2010}:

\begin{equation}
SSIM(a,b) = l(a,b) c(a,b) s(a,b)
\end{equation}

where
\begin{eqnarray}
l(a,b) = \frac{2\mu_a\mu_b+C_1}{\mu_a^2+\mu_b^2+C_1} \label{eqn:luminance_ssim}\\
c(a,b) = \frac{2\sigma_a\sigma_b+C_2}{\sigma_a^2+\sigma_b^2+C_2}\label{eqn:contrast_ssim}\\
s(a,b) = \frac{\sigma_{ab}+C_3}{\sigma_a\sigma_b+C_3}\label{eqn:structure_ssim}
\end{eqnarray}

Equation \ref{eqn:luminance_ssim} measures the similarity in luminance and is equal to 1 only if $\mu_a=\mu_b$.
Similarly, equation \ref{eqn:contrast_ssim} compares the standard deviation of the two images (which corresponds to the contrast). 
It will only equal to 1 if $\sigma_a=\sigma_b$.
The structure comparison equation (\ref{eqn:structure_ssim}) measures the correlation between the images using the covariance $\sigma_{ab}$ between them.

The FSIM is computed using the following equation \citep{Zhang2011a}:

\begin{equation}
\frac{\sum_{x\in\Omega}S_L(\mathbf{x})\cdot PC_m(\mathbf{x})}{{\sum_{x\in\Omega}PC_m(\mathbf{x})}}
\end{equation}

where $\Omega$ is the whole spatial image domain.
$S_L(\mathbf{x})$
%-----------------------------------
%	SUBSECTION 2
%-----------------------------------

\section{Initial CPU Evaluation}
% To exploit parallelizable steps in the algorithm to further enhance suitability on an FPGA
Successive modifications for speed and output quality will be performed until the improvements meet the target +0.1dB increase in quality and 4x speedup.
Initially, these measurements will be performed in the CPU.

The initial algorithm runs will take place on a computer with an Intel Core i7 3632QM Processor 2.2 GHz, 16GB RAM, and NVIDIA GeForce GT 730M 2GB. 
Profiling the algorithm for bottlenecks will be done with the assistance of MATLAB 2015a Profiler. 
This software package can determine the slowest segments of the algorithm and suggest necessary modifications.
MATLAB will also used to code the algorithms and will assist in the preliminary conversion into a form suitable for the Zynq through its Vision HDL Toolbox.
%----------------------------------------------------------------------------------------
%	SECTION 2
%----------------------------------------------------------------------------------------

\section{FPGA Deployment and Comparative Tests}
Once the criteria are met in the CPU version of the algorithm, the next step is to parallelize using the available FPGA resources.
The platform to be used is the Xilinx Zynq-7000 SoC (embedded CPU+FPGA) on a Digilent Zedboard.
The hybrid FPGA and low-power CPU chip enables the dual advantage of quick sequential processing and fine-grained parallel computing.
The Xilinx Vivado Design Suite 2014.2 software will facilitate the process of converting the CPU-based algorithm to parallel.
In addition, Vivado will be used to fine-tune the algorithm and optimize resource usage in the hardware.




 
%% Chapter 5

\chapter{SUMMARY} % Main chapter title

\label{Chapter5} % Change X to a consecutive number; for referencing this chapter elsewhere, use \ref{ChapterX}

\lhead{Chapter 5. \emph{SUMMARY}} % Change X to a consecutive number; this is for the header on each page - perhaps a shortened title

%----------------------------------------------------------------------------------------
%	SECTION 1
%---------------------------------------------------------------------------------------- 
%\input{Chapters/Chapter6} 
%\input{Chapters/Chapter7} 

%----------------------------------------------------------------------------------------
%	THESIS CONTENT - APPENDICES
%----------------------------------------------------------------------------------------

\addtocontents{toc}{\vspace{2em}} % Add a gap in the Contents, for aesthetics

\appendix % Cue to tell LaTeX that the following 'chapters' are Appendices

% Include the appendices of the thesis as separate files from the Appendices folder
% Uncomment the lines as you write the Appendices

% Appendix A

\chapter{Gantt Chart} % Main appendix title

\label{AppendixA} % For referencing this appendix elsewhere, use \ref{AppendixA}

\lhead{Appendix A. \emph{Gantt Chart}} % This is for the header on each page - perhaps a shortened title

% Write your Appendix content here.

% Either use MSProj Gantt Chart, or LaTeX Gantt Chart
% Appendix A

\chapter{Proposed Budget} % Main appendix title

\label{AppendixB} % For referencing this appendix elsewhere, use \ref{AppendixA}

\lhead{Appendix B. \emph{Proposed Budget}} % This is for the header on each page - perhaps a shortened title

Write your Appendix content here.
%\input{Appendices/AppendixC}

\addtocontents{toc}{\vspace{2em}} % Add a gap in the Contents, for aesthetics

\backmatter

%----------------------------------------------------------------------------------------
%	BIBLIOGRAPHY
%----------------------------------------------------------------------------------------

\label{References}

\lhead{\emph{References}} % Change the page header to say "Bibliography"

%\bibliographystyle{unsrtnat} % Use the "unsrtnat" BibTeX style for formatting the Bibliography
\bibliographystyle{apacite}
\bibliography{Bibliographies/SR_image,Bibliographies/HPC_CS,Bibliographies/SR_video,Bibliographies/HPC_SR,Bibliographies/TV_Standards,Bibliographies/Img_Proc,Bibliographies/Bibliography,Bibliographies/FPGA_ImgProc,Bibliographies/FPGA,Bibliographies/HPC_etc,Bibliographies/CS_DL,Bibliographies/CS} % The references (bibliography) information are stored in the file named "Bibliography.bib"

\end{document}  